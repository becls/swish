% Copyright 2018 Beckman Coulter, Inc.
%
% Permission is hereby granted, free of charge, to any person
% obtaining a copy of this software and associated documentation files
% (the "Software"), to deal in the Software without restriction,
% including without limitation the rights to use, copy, modify, merge,
% publish, distribute, sublicense, and/or sell copies of the Software,
% and to permit persons to whom the Software is furnished to do so,
% subject to the following conditions:
%
% The above copyright notice and this permission notice shall be
% included in all copies or substantial portions of the Software.
%
% THE SOFTWARE IS PROVIDED "AS IS", WITHOUT WARRANTY OF ANY KIND,
% EXPRESS OR IMPLIED, INCLUDING BUT NOT LIMITED TO THE WARRANTIES OF
% MERCHANTABILITY, FITNESS FOR A PARTICULAR PURPOSE AND
% NONINFRINGEMENT. IN NO EVENT SHALL THE AUTHORS OR COPYRIGHT HOLDERS
% BE LIABLE FOR ANY CLAIM, DAMAGES OR OTHER LIABILITY, WHETHER IN AN
% ACTION OF CONTRACT, TORT OR OTHERWISE, ARISING FROM, OUT OF OR IN
% CONNECTION WITH THE SOFTWARE OR THE USE OR OTHER DEALINGS IN THE
% SOFTWARE.

\chapter {Operating System Interface}\label{chap:osi}

\section {Introduction}

This chapter describes the operating system interface. Swish is
written in Chez Scheme and runs on Linux, macOS, and Windows. It
provides asynchronous I/O via libuv~\cite{libuv} and database support
via SQLite~\cite{sqlite-guide}.

\section {Theory of Operation}

The operating system interface is written in C99~\cite{c99} as a
shared library that is statically linked against the libuv and SQLite
libraries and dynamically linked against single-threaded Chez Scheme
9.5.  Please refer to Chapter~4 of the \emph{Chez Scheme Version 9
  User's Guide}~\cite{chez-scheme-users-guide} for information on the
foreign function interface.

The single-threaded version of Chez Scheme is used because of its
simplicity. All Scheme code runs in the main thread. In order to keep
this thread responsive, operations that block for more than a
millisecond are performed asynchronously.

For each asynchronous function in the operating system interface, a
Scheme callback procedure is passed as the last argument.  This
callback procedure is later returned to Scheme in a list that includes
the results of the asynchronous function call.

Scheme object locking and unlocking is handled by the operating system
interface because it manages the data structures that contain pointers
to Scheme objects.

The operating system interface uses port objects for files, console
input, pipes to other processes, and TCP/IP connections. A port object
is created by the various open functions, which return a port handle
that is used for read, write, and close operations. Once a port is
closed, its port object is freed.

For interface functions that can fail, an error pair\index{error pair}
\code{(\var{who} . \var{errno})} is returned, where \var{who} is a
symbol representing the name of the particular function that failed
and \var{errno} is either an error number or, in the case of certain
SQLite functions, a pair whose car is the error number and cdr is the
English error string.

Section~\ref{sec:osi-api} describes the programming interface from the
C side. The Scheme library \code{(osi)} provides foreign procedures
for each C function using the same name. For functions that may return
error pair \code{(\var{who} . \var{errno})}, the corresponding Scheme
procedure $p$, e.g., \code{osi\_read\_port}, raises exception
\code{\#(osi-error $p$ \var{who} \var{errno})}. In addition, the
\code{(osi)} library exports another procedure with the \verb|*|
suffix, e.g., \code{osi\_read\_port*}, that returns the error pair.

\section {Programming Interface}\label{sec:osi-api}

Unless otherwise noted, all C strings are encoded in UTF-8.

\subsection {System Functions}

\defineentry{osi\_get\_argv}
\begin{function}
  ptr \code{osi\_get\_argv}(void);
\end{function}

The \code{osi\_get\_argv} function returns a Scheme vector of strings
constructed from the most recent arguments passed to
\code{osi\_set\_argv}.

\defineentry{osi\_get\_bytes\_used}
\begin{function}
  size\_t \code{osi\_get\_bytes\_used}(void);
\end{function}

The \code{osi\_get\_bytes\_used} function returns the number of bytes
used by the C run-time heap. On Linux, it calls the \code{mallinfo}
function. On macOS, it calls the \code{mstats} function. On Windows,
it calls the \code{\_heapwalk} function.

\defineentry{osi\_get\_callbacks}
\begin{function}
  ptr \code{osi\_get\_callbacks}(uint64\_t timeout);
\end{function}

The \code{osi\_get\_callbacks} function returns a list of callback
lists in reverse order of time received. When the list is empty, it
blocks up to \var{timeout} milliseconds before returning. Each
callback list has the form \code{(\var{callback} \var{result} \etc)},
where \var{callback} is the callback procedure passed to the
asynchronous function that returned one or more \var{result}s.

\defineentry{osi\_get\_error\_text}
\begin{function}
  const char* \code{osi\_get\_error\_text}(int err);
\end{function}

The \code{osi\_get\_error\_text} function returns the English string
for the given error number.

\defineentry{osi\_get\_hostname}
\begin{function}
  ptr \code{osi\_get\_hostname}(void);
\end{function}

The \code{osi\_get\_hostname} function returns the host name from
\code{uv\_os\_gethostname}.

\defineentry{osi\_get\_hrtime}
\begin{function}
  uint64\_t \code{osi\_get\_hrtime}(void);
\end{function}

The \code{osi\_get\_hrtime} function returns the current
high-resolution real time in nanoseconds from \code{uv\_hrtime}. It is
not related to the time of day and is not subject to clock drift.

\defineentry{osi\_get\_time}
\begin{function}
  uint64\_t \code{osi\_get\_time}(void);
\end{function}

The \code{osi\_get\_time} function returns the current clock time in
milliseconds in UTC since the UNIX epoch January 1, 1970. On Windows,
it calls the \code{GetSystemTimeAsFileTime} function in
\texttt{kernel32.dll}. On all other systems, it calls the
\code{clock\_gettime} function with \code{CLOCK\_REALTIME}.

\defineentry{osi\_is\_tick\_over}
\begin{function}
  int \code{osi\_is\_tick\_over}(void);
\end{function}

The \code{osi\_is\_tick\_over} function returns 1 if the current time
from \code{uv\_hrtime} is greater than or equal to the threshold set
by the most recent call to \code{osi\_set\_tick} and 0 otherwise.

\defineentry{osi\_list\_uv\_handles}
\begin{function}
  ptr \code{osi\_list\_uv\_handles}(void);
\end{function}

The \code{osi\_list\_uv\_handles} function calls \code{uv\_walk} and
returns a list of pairs \code{(\var{handle} . \var{type})}, where
\var{handle} is the address of the \code{uv\_handle\_t} and \var{type}
is the \code{uv\_handle\_type}.

\defineentry{osi\_make\_uuid}
\begin{function}
  ptr \code{osi\_make\_uuid}(void);
\end{function}

The \code{osi\_make\_uuid} function returns a new universally unique
identifier (UUID) as a bytevector. On Windows, it calls the
\code{UuidCreate} function in \texttt{rpcrt4.dll}. On all other
systems, it calls the \code{uuid\_generate} function.

\defineentry{string->uuid}
\begin{procedure}
  \code{(string->uuid \var{s})}
\end{procedure}
\returns{} a UUID bytevector

The \code{string->uuid} procedure returns the bytevector \var{uuid}
for string \var{s} such that \code{(uuid->string \var{uuid})} is
equivalent to \var{s}, ignoring case. If \var{s} is not a string with
uppercase or lowercase hexadecimal digits and hyphens as shown in
\code{uuid->string}, exception \code{\#(bad-arg string->uuid \var{s})}
is raised.

\defineentry{uuid->string}
\begin{procedure}
  \code{(uuid->string \var{uuid})}
\end{procedure}
\returns{} a string

The \code{uuid->string} procedure returns the uppercase hexadecimal
string representation of \var{uuid},
$\var{HH}_3\var{HH}_2\var{HH}_1\var{HH}_0\code{-}\var{HH}_5\var{HH}_4\code{-}\var{HH}_7\var{HH}_6\code{-}\var{HH}_8\var{HH}_9\code{-}\var{HH}_{10}\var{HH}_{11}\var{HH}_{12}\var{HH}_{13}\var{HH}_{14}\var{HH}_{15}$,
where $\var{HH}_i$ is the 2-character uppercase hexadecimal
representation of the octet at index $i$ of bytevector \var{uuid}.  If
\var{uuid} is not a bytevector of length 16, exception
\code{\#(bad-arg uuid->string \var{uuid})} is raised.

\defineentry{osi\_set\_argv}
\begin{function}
  void \code{osi\_set\_argv}(int argc, const char *argv[]);
\end{function}

The \code{osi\_set\_argv} function stores the \var{argv} pointer to a
C vector of \var{argc} strings for use in the \code{osi\_get\_argv}
function.

\defineentry{osi\_set\_tick}
\begin{function}
  void \code{osi\_set\_tick}(uint64\_t nanoseconds);
\end{function}

The \code{osi\_set\_tick} function sets the threshold for
\code{osi\_is\_tick\_over} to be the current time from
\code{uv\_hrtime} plus the given number of \var{nanoseconds}.

